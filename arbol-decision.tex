\documentclass{article}
\usepackage[spanish]{babel}
\usepackage{graphicx}
\usepackage{amsmath}
\usepackage{listings}
\usepackage{xcolor}

\lstset{
    language=Python,
    basicstyle=\ttfamily\small,
    keywordstyle=\color{blue},
    commentstyle=\color{green},
    stringstyle=\color{red},
    frame=single,
    breaklines=true
}

\title{Actividad Árbol de Decisión}
\author{Tu Nombre}
\date{\today}

\begin{document}

\maketitle

\section{Introducción}
Un \textbf{Árbol de Decisión} es un algoritmo de aprendizaje supervisado que:
\begin{itemize}
    \item Divide recursivamente los datos en subconjuntos mediante reglas basadas en características.
    \item Es interpretable visualmente (flujo tipo diagrama).
    \item Maneja datos categóricos y numéricos.
\end{itemize}
En este ejercicio predecimos si un artista alcanzará el \#1 en Billboard usando atributos musicales.

\section{Metodología}
\subsection{Pasos realizados}
\begin{enumerate}
    \item Carga y limpieza de datos: Codificación de categorías y manejo de valores nulos.
    \item Balanceo implícito con \texttt{class\_weight='balanced'}.
    \item Entrenamiento con profundidad máxima=3 para evitar overfitting.
    \item Evaluación con métricas de clasificación binaria.
\end{enumerate}

\subsection{Código clave}
\begin{lstlisting}
# Modelo con balanceo de clases
model = DecisionTreeClassifier(
    max_depth=3,
    class_weight='balanced'
)
\end{lstlisting}

\section{Resultados}
\begin{itemize}
    \item \textbf{Precisión global}: 78\% (mejor en clase mayoritaria "No Top1").
    \item \textbf{Recall para Top1}: 0.65 (detección aceptable de éxitos).
    \item \textbf{Característica más importante}: \texttt{artist\_type} (género del artista).
\end{itemize}



\section{Conclusión}
\begin{itemize}
    \item El árbol identifica patrones claros (ej: artistas masculinos urbanos tienen más probabilidad de éxito).
    \item Limitación: Profundidad reducida sacrifica precisión por interpretabilidad.
    \item Aprendizaje: Los datos desbalanceados requieren ajustes como \texttt{class\_weight}.
\end{itemize}

\end{document}