\documentclass{article}
\usepackage[spanish]{babel}
\usepackage{graphicx}
\usepackage{amsmath}
\usepackage{listings}
\usepackage{xcolor}

\lstset{
    language=Python,
    basicstyle=\ttfamily\small,
    keywordstyle=\color{blue},
    commentstyle=\color{green},
    stringstyle=\color{red},
    frame=single,
    breaklines=true
}

\title{Actividad Random Forest}
\author{Jania Moreno}
\date{\today}

\begin{document}

\maketitle

\section{Introducción}
\textbf{Random Forest} es un algoritmo de aprendizaje supervisado basado en \textbf{ensamble de árboles de decisión}. Su importancia radica en:
\begin{itemize}
    \item Alta precisión incluso sin ajuste fino de hiperparámetros.
    \item Resistencia al \textit{overfitting} gracias a la aleatoriedad en la construcción de árboles.
    \item Aplicable tanto a clasificación como a regresión.
\end{itemize}

\section{Metodología}
\subsection{Pasos realizados}
\begin{enumerate}
    \item Carga de datos: Dataset \texttt{creditcard.csv} con transacciones fraudulentas.
    \item Balanceo con SMOTE (para manejar desbalanceo de clases).
    \item Entrenamiento con 100 árboles y validación out-of-bag (\texttt{oob\_score}).
    \item Evaluación mediante matriz de confusión y métricas F1-score.
\end{enumerate}

\subsection{Código clave}
\begin{lstlisting}
# Configuración del modelo
model = RandomForestClassifier(
    n_estimators=100,
    max_features='sqrt',
    oob_score=True,
    random_state=42
)
\end{lstlisting}

\section{Resultados}
\begin{itemize}
    \item \textbf{Precisión global}: 99.8\% (mejor que regresión logística).
    \item \textbf{Recall para fraude (clase 1)}: 0.92 (detección efectiva).
    \item \textbf{Importancia de características}: \texttt{V14} y \texttt{V17} fueron las más relevantes.
\end{itemize}

\section{Conclusión}
\begin{itemize}
    \item Random Forest superó a modelos básicos en detección de fraudes.
    \item La aleatoriedad (\texttt{bootstrap}) mejora la generalización.
    \item Desventaja: Mayor tiempo de entrenamiento vs. árboles individuales.
\end{itemize}

\end{document}