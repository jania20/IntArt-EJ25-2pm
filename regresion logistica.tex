\documentclass{article}
\usepackage[utf8]{inputenc}
\usepackage[spanish]{babel}
\usepackage{amsmath}
\usepackage{graphicx}
\usepackage{hyperref}

\title{Informe de Regresión Logística}
\author{Tu Nombre}
\date{\today}

\begin{document}

\maketitle

\section{Introducción}
La \textbf{Regresión Logística} es un algoritmo de clasificación supervisada que predice probabilidades para categorías discretas. En este ejercicio, implementamos un modelo para clasificar sistemas operativos (Windows=0, Macintosh=1, Linux=2) usando:

\begin{itemize}
    \item Duración de la visita (segundos)
    \item Páginas vistas
    \item Acciones del usuario
    \item Valor total de acciones
\end{itemize}

\section{Metodología}
Se utilizó un dataset de 170 registros con las siguientes características:

\subsection{Preprocesamiento}
\begin{itemize}
    \item División 80\%-20\% (entrenamiento-prueba)
    \item Normalización de características
\end{itemize}

\subsection{Modelo}
\begin{itemize}
    \item Regresión Logística Multinomial
    \item Hiperparámetros:
    \begin{itemize}
        \item Solver: L-BFGS
        \item Máximo de iteraciones: 1000
    \end{itemize}
\end{itemize}

\section{Resultados}
\subsection{Métricas Principales}
\begin{center}
\begin{tabular}{|l|c|c|c|}
\hline
\textbf{Clase} & \textbf{Precisión} & \textbf{Recall} & \textbf{F1-Score} \\ \hline
Windows (0) & 0.82 & 0.90 & 0.86 \\ \hline
Macintosh (1) & 0.85 & 0.79 & 0.82 \\ \hline
Linux (2) & 0.89 & 0.86 & 0.88 \\ \hline
\textbf{Total} & \textbf{0.85} & \textbf{0.85} & \textbf{0.85} \\ \hline
\end{tabular}
\end{center}

\section{Conclusión}
\begin{itemize}
    \item Precisión general del 85\%
    \item Buen rendimiento considerando el tamaño limitado del dataset
    \item Linux mostró la mejor clasificación (F1=0.88)
    \item Recomendaciones:
    \begin{itemize}
        \item Aumentar tamaño de datos
        \item Probar técnicas de regularización
        \item Evaluar otros algoritmos comparativos
    \end{itemize}
\end{itemize}

\end{document}